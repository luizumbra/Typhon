%%
% Dúvidas, Projeto Typhon
% =========================
%
% Etapa:
% =====
%   Etapa onde serão geradas as dúvidas e as respostas para o desenvolvimento do
% artigo.
%
% Desenvolvedor:
% =============
%   Luiz Antonio Marques Ferreira.
%
% Data:
% ====
% 09/03/2016
%%

\newcommand{\q}[1] { % \q para questão
  \begin{itemize}
  \item {\color{blue} #1}
  \end{itemize}
}

\section{Duvida}

\q{Qual seria a definição de comunicação?}

Em geral a comunicação é a intenção de troca de informação através da produção e percepção de sinais retirados de um sistema compartilhado de sinais convencionais \cite{russell:2009}.

\q{Como os seres-humanos se comunicam?}

Através de \textbf{linguagens}, que nada mais é do que uma estrutura de sistema de sinais \cite{russell:2009}.

Esse sistema de sinais é conhecido de forma geral como \textbf{linguagens-naturais}.

A ação da comunicação, da conversa, vai além da linguagem falada, portanto, escrever, e sinais de fumação também são usadas como comunicação \cite{russell:2009}.

\q{O que é uma linguagem-natural (NL)?}

É basicamente a linguagem de comunicação humana.

\textbf{Exemplo:} Portugues; Inglês; Francês...

\q{O que é um processamento de linguagem-natural (NLP)?}

Compreender a linguagem humana ao ponto de, pelo menos, os agentes inteligentes serem capazes de nos reaponder \cite{bird:2009}.

\q{Quais são as razões para existirem NLP?}

Existem duas Razões Principais \cite{russell:2009}:

I - Para se Comunicar com os seres humanos.

II - Para adquirir informação a partir da linguagem escrita.

\q{Quantas funcionalidade pode ter um problema NLP?}

\q{Quais são os desafios de um problema NLP?}

\q{Existe o problema de reconhecimento, ou seja, se houver uma entrada de uma frase ambigua, como determinar a ação da comunicação?}

\q{Quais são os desafios possíveis de serem resolvidos por NLP?}

\q{Quais são os desafios, ainda não resolviveis, por um NLP?}

\q{Com relação a frases ambiguas, como determinar a ação do receptor da comunicação?}

Através de implicações lógica.

\q{Como se define uma linguagem formal?}

É um conjunto de palavras, onde cada uma é um símbulo terminal.

\q{O que pode ser usado para analise léxica de uma determinada língua?}

Normalmente são usadas \textbf{Corpora}.

\q{O que é um Corpora?}

Um Corpora é um conjunto de dados linguísticos \cite{bird:2009}

\q{Quais são os corporas mais conhecidos e úteis?}

\q{Como acessar os Corpora?}

Via web.

Usando bibliotecas específicas como o NLTK do Python \cite{bird:2009}

\q{Quais são os tipos básicos para um corpus?}

Essas respostas foram reteridas do livro de \cite{bird:2009}:
\begin{itemize}
\item[I] Isolated
\item[II] Categorized
\item[III] Overlapping
\item[IV] Temporal
\end{itemize}

\q{Máquinas podem pensar?}

Essa pergunta só pode ser respondida pelo Teste de Turing \cite{turing:1950}.

É preciso descobrir a linha que separa o mundo físico e a capacidade intelectual humana.

\q{Qual a definição de ``máquina''?}
  
\q{Qual a definição de ``pensamento''?}
